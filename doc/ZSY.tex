\def\changemargin#1#2{\list{}{\rightmargin#2\leftmargin#1}\item[]}
\let\endchangemargin=\endlist


\documentclass[]{article}
\usepackage{listings}
\usepackage{color}
\usepackage{lscape}
\usepackage[left=3cm,right=2cm,top=2.5cm,bottom=2cm]{geometry}
\usepackage{pdflscape}
\usepackage{multicol}
\usepackage [english]{babel}
\usepackage [autostyle, english = american]{csquotes}
\MakeOuterQuote{"}

%opening
\title{\^{}ZSY: System Status for GT.M/YottaDB}
\author{Sam Habiel, Pharm.D.}

\begin{document}

\maketitle

\begin{abstract}
This paper describes the updates and new functionality in the System Status tools for GT.M/YottaDB.
\end{abstract}

\section{Introduction}
The ZSY Routine of unknown provenance -- was in unreleased VA patch XU*8.0*349 and thus perhaps in the public domain. Its original purpose was to provide an emulation of \%SS in Intersystems Cache. It has been almost entirely re-written and much new functionality has been introduced. While developed with VistA/RPMS in mind, it can be used on any GT.M/YottaDB system with a \$ZINTERRUPT of "I \$\$JOBEXAM\^{}ZSY(\$ZPOS)"

\section{Invocation}

Enter \texttt{D \^{}ZSY} on direct mode. This by default sorts by Process ID. You can use \texttt{D QUERY\^{}ZSY} to query by cpu time.

\section{Output}
The output changes based on the size of the screen: $\leq$ 80 columns; 80 columns--130 columns; and $>$130 columns.

\definecolor{light-gray}{gray}{0.95}

\lstset{ 
	backgroundcolor=\color{light-gray},   % choose the background color; you must add \usepackage{color} or \usepackage{xcolor}; should come as last argument
	basicstyle=\footnotesize\ttfamily,        % the size of the fonts that are used for the code
	breakatwhitespace=false,         % sets if automatic breaks should only happen at whitespace
	breaklines=false,                 % sets automatic line breaking
	captionpos=b,                    % sets the caption-position to bottom
	keepspaces=true,                 % keeps spaces in text, useful for keeping indentation of code (possibly needs columns=flexible)
	rulecolor=\color{black},         % if not set, the frame-color may be changed on line-breaks within not-black text (e.g. comments (green here))
	showspaces=false,                % show spaces everywhere adding particular underscores; it overrides 'showstringspaces'
	showstringspaces=false,          % underline spaces within strings only
	showtabs=false,                  % show tabs within strings adding particular underscores
	caption=80 column output
}

\begin{lstlisting}
GT.M System Status users on 24-APR-18 16:03:06 - (stats reflect accessing 
DEFAULT region ONLY except *)
PID   PName   Device       Routine            Name                CPU Time
4257  mumps   /dev/pts/1   INTRPTALL+8^ZSY                        00:00:00
7369  mumps   BG-0         NEWJOB+5^%ZTM      Taskman ROU 1       00:00:00
7373  mumps   BG-0         GO+26^XMKPLQ       WVEHR,PATCH INSTALL 00:00:00
7391  mumps   BG-0         GO+12^XMTDT        WVEHR,PATCH INSTALL 00:00:00
7397  mumps   BG-0         GETTASK+3^%ZTMS1                       00:00:00
7403  mumps   BG-0         GETTASK+3^%ZTMS1   Sub 7403            00:00:00
7409  mumps   BG-0         I6+2^%ZTM                              00:00:00
7411  mumps   BG-0         GETTASK+3^%ZTMS1   Sub 7411            00:00:00
\end{lstlisting}
%\end{changemargin}


\begin{landscape}
\lstset{
	caption=130 column output
}
\begin{lstlisting}
GT.M System Status users on 24-APR-18 16:35:06 - (stats reflect accessing DEFAULT region ONLY except *)
PID   PName   Device       Routine            Name                CPU Time OP/READ   NTR/NTW        NR0123    #L   %LSUCC  %CFAIL
4257  mumps   /dev/pts/1   INTRPTALL+8^ZSY                        00:00:00 1.14      10/0           0/0/0/0   1    0/0     0/7
7369  mumps   BG-0         IDLE+3^%ZTM        Taskman ROU 1       00:00:00 493.05    10k/0k         1/0/0/0   0    99.95%  0/426
7373  mumps   BG-0         GO+26^XMKPLQ       WVEHR,PATCH INSTALL 00:00:00 90.14     1905/12        0/0/0/0   2    9/14    0/33
7391  mumps   BG-0         GO+12^XMTDT        WVEHR,PATCH INSTALL 00:00:00 59.73     669/12         0/0/0/0   2    9/9     0/23
7397  mumps   BG-0         GETTASK+3^%ZTMS1                       00:00:00 784.04    19k/2k         1/0/0/0   1    99.96%  0/1977
7403  mumps   BG-0         GETTASK+3^%ZTMS1   Sub 7403            00:00:00 0.00      9766/968       3/0/0/0   1    99.35%  0/973
7411  mumps   BG-0         GETTASK+3^%ZTMS1   Sub 7411            00:00:00 0.00      9738/965       3/1/0/0   1    99.53%  0/970
7419  mumps   BG-0         GETTASK+3^%ZTMS1   Sub 7419            00:00:00 0.00      9717/963       7/3/0/0   1    99.02%  0/967
7422  mumps   BG-0         GETTASK+3^%ZTMS1   Sub 7422            00:00:00 0.00      9678/959       13/0/0/0  1    98.25%  0/970
\end{lstlisting}
\lstset{
	caption=$>$130 column output
}
\begin{changemargin}{-0.25in}{-0.25in}
\begin{lstlisting}
GT.M System Status users on 24-APR-18 17:05:25 - (stats reflect accessing DEFAULT region ONLY except *)
PID   PName   Device       Routine            Name                CPU Time OP/READ   NTR/NTW     NR0123    #L   %LSUCC  %CFAIL R MB*   W MB* SP MB*
4257  mumps   /dev/pts/1   INTRPTALL+8^ZSY                        00:00:00 1.14      10/0        0/0/0/0   1    0/0     0/7    5.66    12.98 0.10
7369  mumps   BG-0         IDLE+3^%ZTM        Taskman ROU 1       00:00:00 955.95    20k/1k      3/0/0/0   0    99.80%  0/721  2.77    0.68  0.10
7373  mumps   BG-0         GO+26^XMKPLQ       WVEHR,PATCH INSTALL 00:00:00 166.86    3516/12     0/0/0/0   2    9/14    0/33   3.21    0.03  0.10
7391  mumps   BG-0         GO+12^XMTDT        WVEHR,PATCH INSTALL 00:00:00 100.64    1119/12     0/0/0/0   2    9/9     0/23   0.23    0.00  0.10
7397  mumps   BG-0         GETTASK+3^%ZTMS1                       00:00:01 1505.24   37k/4k      2/0/0/0   1    99.96%  81.36% 0.29    2.51  0.10
7403  mumps   BG-0         GETTASK+3^%ZTMS1   Sub 7403            00:00:00 0.00      18k/2k      8/0/0/0   1    99.47%  89.76% 0.00    0.77  0.10
7411  mumps   BG-0         GETTASK+3^%ZTMS1   Sub 7411            00:00:00 0.00      18k/2k      3/1/0/0   1    99.67%  89.79% 0.00    0.73  0.10
7419  mumps   BG-0         GETTASK+3^%ZTMS1   Sub 7419            00:00:00 0.00      18k/2k      19/4/0/0  1    99.09%  89.80% 0.00    0.79  0.10
7422  mumps   BG-0         GETTASK+3^%ZTMS1   Sub 7422            00:00:00 0.00      18k/2k      16/0/0/0  1    98.50%  89.78% 0.00    0.41  0.10
\end{lstlisting}
\end{changemargin}


Here is by contrast the Cache system status:

\lstset{
	caption=Cache System Status
}
\begin{lstlisting}
                   Cache System Status:  4:00 pm 02 May 2018
Process  Device      Namespace      Routine         CPU,Glob  Pr User/Location
     92  /dev/null   PLA            XMKPLQ         8415,815   0  %System
     61  |TCP|9430   PLA            %ZISTCPS       2805,187   0  %System
     66  /dev/null   PLA            %ZTM           9180,1078  0  %System
     68  |TCP|8001   PLA            XOBVTCPL       5355,416   0  %System
     93  /dev/null   PLA            XMTDT          6885,380   0  %System
     94  /dev/null   PLA            %ZTMS1        18615,2779  0  %System

6 user, 0 system, 256 mb global/23 mb routine cache
\end{lstlisting}
\end{landscape}
\pagebreak

\begin{multicols}{2}
\section{Explanation of Columns}

\subsection{PID}
This is the Process Number of each process. On GT.M/YottaDB, the process number is the same as job number. This unfortunately is not the same as the Taskman task number.

\subsection{PName}
Process Name. Most of the time it will be "mumps", but it can be "java", "node" or some other process that is opening the database.

\subsection{Device}
Consider this listing:
\lstset{caption=Device}
\begin{lstlisting}
Device
BG-0         
BG-0         
BG-0         
BG-S14823    
/dev/pts/22  
BG-0         
BG-/dev/null
/dev/null
\end{lstlisting}
This shows the currently active device.

The most obvious thing is the "BG". This means that the job is a background job (not launched by a user, but launched originally by a Job command). However, even background jobs can interact with users: The last BG device in this is listing is actually a CPRS session. The process was created using the Job command.

User "Roll and Scroll" sessions will show as /dev/pty/...

Devices with socket listeners will show as BG-Sport, where port is a numeric port number.

BG-0 devices are usually background jobs without devices in Taskman.

Broker jobs originating from xinetd will show a device of /dev/null unless a device is opened.

\subsection{Routine}
This shows the currently executing line.

\subsection{Name}
This is either the name of the user from File 200 if the process has a DUZ variable defined; or process name if defined in \^{}XUTL, or nothing otherwise.

\subsection{CPU Time}
Total CPU time (User + System) as reported by ps.

\subsection{OP/READ}
Operations/Read: Number of global operations per database block read obtained via the following formula:  \[\frac{DTA+GET+ORD+ZPR+QRY}{DRD}\]

DRD is the number of database reads; DTA, GET, ORD, ZPR and QRY are the number of operations on globals of \$DATA, \$GET, \$ORDER, reverse \$ORDER, and \$QUERY (forward and reverse) respectively. This statistic is only for the DEFAULT region of the database.

\subsection{NTR/NTW}
This is the number of non-transaction reads and non-transactions writes into the DEFAULT region of the database. If NTR $>$ 9999, the values will divided by 1024 and displayed in kilos. This statistic is only for the DEFAULT region of the database.

\subsection{NR0123}
This is the number of \# of Non-TP transaction Restarts at try 0, 1, 2, and 3. A large number of retries at 2 or 3 is a bad sign. This statistic is only for the DEFAULT region of the database.

\subsection{\#L}
Number of locks held by the process.

\subsection{\%LSUCC}
Lock success percentage calculated by: \[\frac{LKS}{LKS+LKF}\]

LKS is the number of successful locks; and LKF is the number of unsuccessful locks.  If LKS + LKF is $<$ 100, then it will be shown as a fraction; otherwise it will be shown as a percentage with 2 decimal points. This statistic is only for the DEFAULT region of the database.

\subsection{\%CFAIL}
Critical section acquisition failure percentage calculated by: \[\frac{CFT}{CFT + CAT}\]

CFT is the total of blocked critical section acquisitions; CAT is the total of critical section acquisitions successes. Like locks, if CFT + CAT $<$ 100, then it will be shown as a fraction; otherwise, it will be shown as a percentage with 2 decimal points. This statistic is only for the DEFAULT region of the database.

This number is hard to use in isolation. It is only useful as a trend point. My system shows $>$80\% failure rate for taskman processes in a tight loop.

\subsection{R MB}
Data read from disk in megabytes, as reported by the OS.

\subsection{W MB}
Data written to disk in megabytes, as reported by the OS.

\subsection{SP MB}
This is how large the string pool is. This is a good proxy for how much heap memory is being used by GT.M/YDB for the symbol table. This is obtained using \$view("spsize").

\end{multicols}
\pagebreak

\section{Job Examination Utility}
A job examination utility is provided in order to assist with troubleshooting specific jobs. It can be invoked in two ways: either run D ZJOB\^{}ZSY, and then select a job number from the list; or run D ZJOB\^{}ZSY(job\_number) to interrogate a specific job number. You can also use one of these alternate entry points:

\begin{itemize}
	\item D EXAMJOB\^{}ZSY[(job\_number)]                                                      
	\item D VIEWJOB\^{}ZSY[(job\_number)]                                                       
	\item D JOBVIEW\^{}ZSY[(job\_number)] 
\end{itemize}

Pressing enter on the screens that are displayed updates them. If you want to exit, type \^{}.

Following is the screen obtained from D ZJOB\^{}ZSY:
\lstset{
	caption=D ZJOB\^{}ZSY
}
\begin{lstlisting}

GT.M System Status users on 02-MAY-18 15:16:38
PID   PName   Device       Routine            Name                CPU Time
11422 mumps   BG-0         IDLE+3^%ZTM        Taskman ICARUS 1    00:00:03
11481 mumps   BG-0         GO+26^XMKPLQ       POSTMASTER          00:00:00
11483 mumps   BG-0         GO+12^XMTDT        POSTMASTER          00:00:00
11489 mumps   BG-S14823    LGTM+25^%ZISTCPS   POSTMASTER          00:00:01
13063 mumps   /dev/pts/21  INTRPTALL+8^ZSY    USER,ONE            00:03:01
15287 mumps   BG-0         GETTASK+3^%ZTMS1                       00:00:00

Total 6 users.

Enter a job number to examine (^ to quit): 
\end{lstlisting}

Selecting a job number, or invoking it directly using D ZJOB\^{}ZSY(PID) will let you to the follow screen:

\lstset{
	caption=Job Examination Screen
}
\pagebreak

\begin{lstlisting}
JOB INFORMATION FOR 11483 (2018-MAY-02 15:26:57)
AT: GO+12^XMTDT:  . . H XMHANG

Stack: 
1. SUBMGR+3^%ZTMS1                       D PROCESS^%ZTMS2 G:$D(ZTQUIT) QUIT^%ZTMS
2. PROCESS+9^%ZTMS2                      D TASK^%ZTMS3 I ZTYPE="C"!$D(ZTNONEXT) Q
3. 4+10^%ZTMS3                           D RUN
4. RUN+4^%ZTMS3                          D @ZTRTN
5. GO+5^XMTDT                            F  Q:$P($G(^XMB(1,1,0)),U,16)  D
6. GO+11^XMTDT                           . F  D  Q:$$TSTAMP^XMXUTIL1-XMWAIT>60
7. GO+12^XMTDT:                          . . H XMHANG

Locks: 
LOCK ^XMBPOST("POST_Tickler") LEVEL=1
LOCK ^%ZTSCH("TASK",3479) LEVEL=1

Devices: 
0 OPEN RMS STREAM NOWRAP 
0-out OPEN RMS STREAM NOWRAP 

Breakpoints: 

Global Stats for default region: 
BTD: 0         CAT: 26        CFE: 0         CFS: 16k       CFT: 129       CQS: 0 
CQT: 0         CTN: 188m      CYS: 1         CYT: 1         DEX: 0         DFL: 0 
DFS: 0         DRD: 13        DTA: 30        DWT: 14        GET: 2262      JBB: 1568 
JEX: 0         JFB: 0         JFL: 0         JFS: 0         JFW: 0         JRE: 0 
JRI: 0         JRL: 12        JRO: 1         JRP: 0         KIL: 4         LKF: 2 
LKS: 9         NBR: 10k       NBW: 11        NR0: 0         NR1: 0         NR2: 0 
NR3: 0         NTR: 3985      NTW: 12        ORD: 1677      QRY: 0         REG: DEFAULT 
SET: 10        TBR: 0         TBW: 0         TC0: 0         TC1: 0         TC2: 0 
TC3: 0         TC4: 0         TR0: 0         TR1: 0         TR2: 0         TR3: 0 
TR4: 0         TRB: 0         TTR: 0         TTW: 0         ZPR: 4         ZTR: 0 


String Pool (size,currently used,freed): 102056,5684,0

Enter to Refersh, V for variables, I for ISVs, K to kill
L to load variables into your ST and quit, ^ to go back: 
D to debug (broken), Z to zshow all data for debugging.
\end{lstlisting}

Pressing enter on this page refreshes the information. Pressing enter frequently will give you a good view of what's happening in an execution.

The sections should be self-explanatory.

Pressing V or I will show you variables and Intrinsic Special Variables(ISV). They are just printed in a vertical list so that they are easy to copy and paste.
\lstset{
	caption=Variable List
}
\begin{lstlisting}
%=1
%ZPOS="GO+12^XMTDT"
%ZTIME=5596168159
%ZTPFLG("BalLimit")=100
%ZTPFLG("HOME")="ICARUS:foia.2018.02"
%ZTPFLG("LOCKTM")=3
%ZTPFLG("MIN")=1
%ZTPFLG("RT")=0
%ZTPFLG("USER")=18
%ZTPFLG("XUSCNT")=59
%ZTPFLG("ZTPN")=1
%ZTPFLG("ZTREQ")=1
%reference="^XMB(3.9,""AF"",0)"
DILOCKTM=3
DT=3180502
DTIME=1
DUZ=.5
DUZ(0)="@" (etc...)
\end{lstlisting}

\lstset{
	caption=ISV List
}
\begin{lstlisting}
ISVs: 
$DEVICE=""
$ECODE=""
$ESTACK=9
$ETRAP="D ERROR^%ZTMS HALT"
$HOROLOG="64770,56506"
$IO=0
$JOB=11483
$KEY=""
$PRINCIPAL=0
$QUIT=1
$REFERENCE="^XUTL(""XUSYS"",11483,""JE"",""I"",10)"
$STACK=9
$STORAGE=2147483647
$SYSTEM="47,foia.2018.02"
$TEST=1
$TLEVEL=0 (etc...)
\end{lstlisting}

K will kill the process (actually, it sends it to HALT\^{}ZU after cleaning it up--killing it softly by forcing it to drink a cup). \^{} will take you back; D (debug) is currently broken; and Z is there for the developer's use.

\pagebreak

\section{Other Entry Points}
\^{}ZSY includes various other entry points that can be used by developers and system managers. Here they are.

\subsection{D TMMGR\^{}ZSY [Public]}
List Taskman Manager processes only. Output:
\lstset{
	caption=TMMGR Entry Point
}
\begin{lstlisting}
GT.M System Status users on 02-MAY-18 16:01:02
PID   PName   Device       Routine            Name                CPU Time
11422 mumps   BG-0         IDLE+3^%ZTM        Taskman ICARUS 1    00:00:04
\end{lstlisting}

\subsection{D TMSUB\^{}ZSY [Public]}
List Taskman Submanger processes, including those currently "otherwise" engaged.
\lstset{
	caption=TMSUB Entry Point
}
\begin{lstlisting}
GT.M System Status users on 02-MAY-18 16:03:46
PID   PName   Device       Routine            Name                CPU Time
11481 mumps   BG-0         GO+26^XMKPLQ       POSTMASTER          00:00:00
11483 mumps   BG-0         GO+12^XMTDT        POSTMASTER          00:00:00
11489 mumps   BG-S14823    LGTM+25^%ZISTCPS   POSTMASTER          00:00:01
15287 mumps   BG-0         GETTASK+3^%ZTMS1                       00:00:02
\end{lstlisting}

\subsection{\$\$UNIXLSOF\^{}ZSY(.procs) [Kernel Use Only]}
This gives you a listing of all the processes accessing the DEFAULT region of the database. The extrinsic output is the number of processes; while \texttt{.procs} contains an M array of the process numbers.

This API should be used by Kernel level applications only.
\lstset{
	caption=UNIXLSOF Entry Point
}
\begin{lstlisting}
>W $$UNIXLSOF^ZSY(.zzz)
6
>zwrite zzz
zzz(1)=11422
zzz(2)=11481
zzz(3)=11483
zzz(4)=11489
zzz(5)=13063
zzz(6)=15287
\end{lstlisting}

\subsection{D INTRPT\^{}ZSY(PID) [Kernel Use Only]}
Send a GT.M Interrupt to a process specified by its PID.

\subsection{D INTRPTALL\^{}ZSY[(.procs)] [Kernel Use Only]}
Use the \$\$UNIXLSOF\^{}ZSY(.procs) API to find all processes accessing the DEFAULT region and interrupt all of them. You can optionally pass in .procs to get a list of all the PIDs that got interrupted.

\subsection{D HALTALL\^{}ZSY [Kernel Use Only]}
"Softly" (\^{}XUSCLEAN, HALT\^{}ZU) kill all processes.

\subsection{D HALTONE\^{}ZSY(PID) [Kernel Use Only]}
"Softly" kill a single process specified by PID

\end{document}

